\documentclass{article}
\usepackage{graphicx} % Required for inserting images
\usepackage{chngcntr}
\counterwithin{figure}{section}

\title{Hop-Tales}
\author{Enrico Casadei, 
\texttt{enrico.casadei18@studio.unibo.it}
\and
Simone Capacci,
\texttt{simone.capacci3@studio.unibo.it}
\and
Simone Gentili,
\texttt{simone.gentili4@studio.unibo.it}
\and
Eleonora Bartoletti,
\texttt{eleonora.bartoletti2@studio.unibo.it}
}
\date{February 2026}

\renewcommand{\contentsname}{Indice} % cambia "Contents" in "Indice"

\begin{document}

\maketitle

\newpage

\tableofcontents  % genera l’indice

\section{Analisi}
\subsection{Descrizione e requisiti}
Il gruppo si pone come obiettivo la realizzazione di un gioco platformer chiamato Hop-Tales. Il gioco si sviluppa in 3 livelli: nel primo il giocatore si muove in avanti evitando i nemici fino a raggiungere un castello posto alla fine del livello; nel secondo livello, simile al primo, si utilizza un secondo personaggio che a fine livello raggiunge lo stesso castello; nel terzo livello, da giocare in due, il gioco diventa un puzzle-platformer, in cui i due personaggi devono risolvere dei mini-puzzle per finalmente riunirsi.
\paragraph{Requisiti funzionali}
\begin{itemize}
    \item Il giocatore sarà in grado di muoversi liberamente su due dimensioni per tutti e tre i livelli.
    \item I primi due livelli avranno lo stesso stile da classico platformer, in cui il giocatore procede verso una direzione finchè non completa il livello, mentre il terzo livello sarà composto da una stanza predefinita contenente dei puzzle.
    \item Il giocatore avrà la possibilità di interagire con diversi elementi del livello che avranno particolari effetti; in particolare nei primi due livelli forniranno un potenziamento al personaggio, mentre nel terzo livello permetteranno di risolvere i puzzle al fine di completare il livello.
    \item Tra gli elementi interagibili nei vari livelli ci saranno anche le monete, oggetti raccoglibili che potranno essere utilizzati nel menu per fini puramente decorativi.
    \item Sarà presente un menu principale da cui saranno accessibili i vari livelli oltre allo shop, in cui si potranno spendere le monete raccolte per comprare delle skin per i personaggi, e alle impostazioni, che riguarderanno principalmente effetti audio.
    \item Saranno presenti diversi tipi di nemici in grado di ostacolare il giocatore. Essi avranno anche la possibilità di "terminare" la partita, colpendo il giocatore un certo numero di volte.    
\end{itemize}
\paragraph{Requisiti non funzionali}
\begin{itemize}
    \item Il gioco dovrà funzionare senza errori su diversi sistemi operativi, tra cui sicuramente una qualsiasi distribuzione Linux. 
\end{itemize}
\subsection{Modello del Dominio}
All'avvio del gioco si crea una finestra dove viene mostrato il menu principale, da cui il giocatore può scegliere di aprire lo shop (per comprare delle skin per i personaggi), le opzioni (per modificare gli effetti audio) o giocare. Scegliendo l'ultima opzione a sua volta gli verrà chiesto quale livello vuole affrontare. In seguito la finestra mostrerà il livello selezionato e all'interno di esso si troveranno diversi elementi: ovviamente ci sarà il personaggio del giocatore, in grado di muoversi liberamente, saltare e attaccare se provvisto di power-up, inoltre saranno presenti dei nemici, anch'essi liberi di muoversi (anche se in maniera predefinita) e attaccare. Le due entità condividono 
i punti vita, ovvero un indice numerico che si abbassa ogni volta che l'entità subisce danni. Qualora l'indice scenda a 0 o inferiore nel caso del giocatore finisce la partita, mentre per quanto riguarda i nemici essi "muoiono".
Oltre alle entità saranno presenti diversi oggetti che potranno essere tangibili, interagibili o entrambi. Tra i tangibili ci saranno i semplici blocchi usati per modellare il livello, i blocchi contenenti i power-up, i power-up stessi, le monete e gli oggetti relativi ai puzzle, che ovviamente saranno anche interagibili. Gli elementi sopracitati con le loro relazioni sono riassunti nella Figura~\ref{fig:uml_analisi}. 
\begin{figure}[htbp]
    \centering
    \includegraphics[width=8cm]{Mermaid Chart - Create complex, visual diagrams with text.-2026-01-17-104839.png}
    \caption{Diagramma UML dell'analisi del dominio}
    \label{fig:uml_analisi}
\end{figure}

\section{Design}
Blablabla.

\section{Sviluppo}
Blablabla.

\section{Commenti finali}
Blablabla.


\end{document}
